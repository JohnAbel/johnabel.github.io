%
% abel cv
% format duplicated from Liana F Lareau, UC Berkeley
%

\documentclass[line,10pt]{res}

% fonts
%\usepackage[T1]{fontenc}
%\usepackage{lmodern}
%\usepackage[sc]{mathpazo}
%\usepackage{textcomp}

\usepackage{fontspec}
\setsansfont[Ligatures=TeX,
             BoldFont={Avenir Heavy}]{Avenir Light}
\renewcommand{\familydefault}{\sfdefault}

% other packages
\usepackage{doi}

% if using margin, comment this out
\usepackage[top=1in, bottom=1in, left=1in, right=1in, letterpaper]{geometry}
\setlength{\resumewidth}{6.5in}
\newsectionwidth{0em}



%% if not using margin, comment this out
%\oddsidemargin -.5in
%\evensidemargin -.5in
%\textwidth=6.0in
%\itemsep=0in
%\parsep=0in
% if using pdflatex:
%\setlength{\pdfpagewidth}{\paperwidth}
%\setlength{\pdfpageheight}{\paperheight} 



\begin{document}

\null\hspace{0in}{\large\bf JOHN H ABEL}\\[0.5em]
\null\hspace{0.5em}{Massachusetts General Hospital, Harvard Medical School}\\
\null\hspace{0.5em}{Massachusetts Institute of Technology}\\
\null\hspace{0.5em}{abelj (at) mit (dot) edu}\\
\null\hspace{0.5em}{www.neurostat.mit.edu}%\hfill \normalfont{\textit{CV}}}
\vspace{1em}

\begin{resume}

%\section{\bf Research\\ Interests}
%Dynamical systems, systems biology, stochastic simulations, control theory, circadian rhythm
%\vspace*{0.1in}


\section{\bf EDUCATION}
\vspace{1em}
\begin{tabular}{p{1.0in} p{5.5in}}
 2018 & \textbf{Harvard University}\\
      & PhD in Systems Biology\\[0.8em]
 2015 & \textbf{UC Santa Barbara}\\
      & MS in Chemical Engineering\\[0.8em]
 2013 & \textbf{Tufts University}\\
      & BS in Chemical Engineering \textit{magna cum laude}\\
\end{tabular}

\section{\bf RESEARCH POSITIONS} % for post-doc stuff
\vspace{1em}

\begin{tabular}{p{1.0in} p{4.5in} p{1in}}
 Postdoctoral & \textbf{Harvard Medical School} & 2018 - \\
     fellow   & Department of Anesthesia, Critical Care and Pain Medicine & \\
              & Massachusetts General Hospital & \\
              & \textit{with Emery N.\ Brown} & \\[0.8em]
  Research & \textbf{Massachuetts Institute of Technology} & 2018 - \\
 affiliate & Picower Institute for Learning and Memory & \\
 \null & & \\[0.8em]
 Graduate  & \textbf{Harvard University} & 2015 - 2018\\
researcher & Department of Systems Biology & \\
           & \textit{with Francis J.\ Doyle III, Elizabeth B.\ Klerman} & \\[0.8em]
  Graduate & \textbf{UC Santa Barbara} & 2013 - 2015\\
researcher & Department of Chemical Engineering & \\
           & \textit{with Francis J.\ Doyle III, Linda R.\ Petzold} & \\[0.8em]
 Undergraduate & \textbf{Tufts University} & 2011 - 2013\\
  researcher   & Department of Chemical Engineering & \\
               &\textit{with Hyunmin Yi} & \\
\end{tabular}

\section{\bf HONORS, AWARDS, AND FELLOWSHIPS}
\vspace{1em}
\begin{tabular}{p{1.0in} p{5.5in}}
2019 & NIH Postdoctoral F32 Ruth L.\ Kirschstein NRSA Fellowship \\
2018 & NIH Postdoctoral T32 Traineeship, Harvard Medical School \\
2016 & NIH Predoctoral T32 Traineeship, Harvard Medical School \\
2016 & Research Excellence Award, Society for Research of Biological Rhythms\\
%Mellichamp Travel Award \hfill 2015
%2015 & UC Santa Barbara Best Bioengineering Research Presentation Award\\
%2015 & Materials Research Laboratory Education Outreach Award (UC Santa Barbara)\\
2014 & Mellichamp Fellowship in Systems Biology, UC Santa Barbara\\
2014 & Honorable Mention, NSF Graduate Research Fellowship Program\\
2013 & High Thesis Honors, Tufts University\\
2012 & Meritorious Winner, COMAP Mathematical Contest in Modeling\\
2011 & Meritorious Winner, COMAP Mathematical Contest in Modeling\\
2009 & ExxonMobil Teagle Foundation Scholarship\\
\end{tabular}

\section{\bf FUNDING}
\vspace{1em}
Automatic control of propanidid and propofol anesthesia-induced unconsciousness\\ 
NIH/NIA F32-AG064886, role: PI, 9/1/2019-9/1/2021


\section{\bf JOURNAL ARTICLES}
\vspace{0.5em}
https://scholar.google.com/citations?user=B5qvOBcAAAAJ
\vspace{1em}
\begin{enumerate}
    \setlength\itemsep{0.8em}
%    \item \textbf{JH Abel}$^\dagger$, MA Badgeley$^\dagger$, B Meschede-Krasa, G Schamberg, IC Garwood, K Lecamwasam, S Chakravarty, DW Zhou, M Keating, KJ Pavone, PL Purdon, and EN Brown, ``Machine learning of EEG spectra classifies conscious state during propofol-induced anesthesia in healthy volunteers and in the OR,'' submitted.
    \item F Rijo-Ferreira, VA Acosta-Rodriguez$^\dagger$, \textbf{JH Abel}$^\dagger$, I Kornblum, I Bento, G Kilaru, EB Klerman, MM Mota, and JS Takahashi, ``The malaria parasite has an intrinsic clock,'' submitted.
    \item Y~Shan, \textbf{JH Abel}, Y~Li, M~Izumo, KH~Cox, B~Jeong, {S-H}~Yoo DP~Olson, FJ~Doyle~III, and JS~Takahashi, ``Dual-color single-cell imaging of the suprachiasmatic nucleus reveals a circadian role in network synchrony,'' submitted.
        \item \textbf{JH Abel}, K~Lecamwasam, MA St.\ Hilaire, EB~Klerman, ``Recent advances in modeling sleep: from the clinic to society and disease,'' 
    \textit{Current Opinion in Physiology}, 
    15, 2020. \\doi: 10.1016/j.cophys.2019.12.001
    \item \textbf{JH Abel}, A~Chakrabarty, EB Klerman, and FJ Doyle III, ``Pharmaceutical-based entrainment of circadian phase via nonlinear model predictive
	    control,'' \textit{Automatica} 100, 2019. \\doi: %\href{https://doi.org/
    10.1016/j.automatica.2018.11.012%}{[doi]}
    \item C~Mazuski, \textbf{JH Abel}, S~Chen, T~Hermanstyne, FJ~Doyle~III, and ED Herzog, ``Entrainment of circadian rhythms depends on firing rates and neuropeptide release of VIP SCN neurons,'' \textit{Neuron} 99 (3), 2018.\\ doi: %\href{https://dx.doi.org/
    10.1016/j.neuron.2018.06.029%}{[doi]} 
    \item V~Carmona-Alcocer, \textbf{JH Abel}, TC~Sun, LR~Petzold, FJ~Doyle~III, CL~Simms, and ED Herzog, ``Ontogeny of circadian rhythms and synchrony in the suprachiasmatic nucleus,'' \textit{Journal of Neuroscience} 38 (6), 2018. doi: %\href{https://dx.doi.org/
    10.1523/jneurosci.2006-17.2017%}{[doi]} 
    \item \textbf{JH Abel}$^\dagger$, B~Drawert$^\dagger$, A~Hellander, and LR Petzold, ``GillesPy: a Python package for stochastic model building and simulation,'' \textit{IEEE Life Sciences Letters} 2 (3), 2017. doi: %\href{https://dx.doi.org/
    10.1109/lls.2017.2652448%}{[doi]}
    \item \textbf{JH Abel} and FJ~Doyle~III, ``A systems theoretic approach to analysis and control of mammalian circadian dynamics,'' \textit{Chemical Engineering Research and Design} 116, 2016. \\doi: %\href{https://dx.doi.org/
    10.1016/j.cherd.2016.09.033%}{[doi]}
    \item S~Jung, \textbf{JH Abel}, J~Starger, and H~Yi, ``Porosity-tuned chitosan-polyacrylamide hydrogel microspheres for improved protein conjugation,'' \textit{Biomacromolecules} 17 (7), 2016. \\doi:%\href{https://dx.doi.org/
    10.1021/acs.biomac.6b00582%}{[doi]}
    \item \textbf{JH Abel}$^\dagger$, K~Meeker$^\dagger$, D~Granados-Fuentes, PC~St.~John, T~Wang, BB~Bales, FJ~Doyle~III, ED~Herzog, and LR~Petzold,
        ``Functional network inference of the suprachiasmatic nucleus,''  \textit{Proceedings of the National Academy of Sciences} 113 (16), 2016. doi:  %\href{https://dx.doi.org/
10.1073/pnas.1521178113%}{[doi]}
    \item E~Kang, S~Jung, \textbf{JH Abel}, A~Pine, and H~Yi, ``Shape-encoded chitosan-polyacrylamide hybrid hydrogel microparticles with controlled macroporous structures via replica molding for programmable biomacromolecular conjugation,'' \textit{Langmuir} 32 (21), 2016. doi: %\href{https://dx.doi.org/
    10.1021/acs.langmuir.5b04653%}{[doi]}
    \item \textbf{JH Abel}, LA~Widmer, PC~St.~John, J~Stelling, and FJ~Doyle~III,
        ``A coupled stochastic model explains differences in Cry knockout behavior,'' \textit{IEEE Life Sciences Letters} 1 (1), 2015. \\doi: %\href{https://dx.doi.org/
10.1109/lls.2015.2439498%}{[doi]}
    \item PC~St.~John, SR~Taylor, \textbf{JH Abel}, and FJ~Doyle~III, ``Amplitude metrics for cellular circadian bioluminescence reporters,'' \textit{Biophysical Journal} 107 (11), 2014. doi: %\href{https://dx.doi.org/10.1016/
    j.bpj.2014.10.026%}{[doi]}
\end{enumerate}

\section{\bf PEER-REVIEWED CONFERENCE PROCEEDINGS}
\vspace{1em}\null
\begin{enumerate}
    \setlength\itemsep{0.8em}
    \item \textbf{JH Abel}, MA Badgeley, TE Baum, S Chakravarty, PL Purdon, EN Brown, ``Constructing a control-ready model of EEG signal during general anesthesia in humans,'' in press for \textit{21st IFAC World Congress}. arXiv:~1912.08144
    \item AS Waite$^\dag$, S Chakravarty$^\dag$, \textbf{JH Abel}, EN Brown, ``A simulation-based comparative analysis of PID and LQG control for closed-loop anesthesia delivery,'' in press for \textit{21st IFAC World Congress}.\\arXiv:~1912.06724
    \item \textbf{JH Abel}, A~Chakrabarty, and FJ~Doyle~III, ``Nonlinear model predictive control for circadian entrainment using small-molecule pharmaceuticals,'' \textit{Proceedings of the 20th IFAC World Congress}, July 2017. doi: %\href{https://dx.doi.org/
    10.1016/j.ifacol.2017.08.1596%}{[doi]}
\end{enumerate}

\section{\bf BOOK CHAPTERS}
\vspace{1em}\null
\begin{enumerate}
    \setlength\itemsep{0.8em}
    \item \textbf{JH Abel}, A~Chakrabarty, and FJ~Doyle~III, ``Controlling time: nonlinear model predictive control for populations of circadian oscillators,'' in \textit{Emerging Applications of Control and System Theory}, R~Tempo, S~Yurkovich, P~Misra Eds. New York, NY: Springer Publishing, 2018. ISBN: 978-3-319-67068-3
    \item B~Drawert, K~Sanft, \textbf{JH Abel}, S~Hellander, A~Pourzanjani, A~Hellander, and LR~Petzold, ``Simulation of well-mixed and spatially inhomogeneous biochemical systems,'' in \textit{Quantitative Biology: Theory, Computational Methods, and Models},
B~Munsky, W~Hlavacek, L~Tsimring, Eds.
Cambridge, MA: The MIT Press, 2018. ISBN: 978-0-262-03808-9
\end{enumerate}

\section{\bf PATENTS AND APPLICATIONS}
\vspace{1em}\null
\begin{enumerate}
    \setlength\itemsep{0.8em}
    \item H~Yi, S~Jung, and \textbf{JH Abel} ``Macroporous chitosan-polyacrylamide hydrogel microspheres and preparation thereof,'' US Patent App.\ 16/311,063, published 2019.
    \item H~Yi, E~Kang, S~Jung, and \textbf{JH Abel}, ``Fabrication of macroporous polymeric hydrogel microparticles,'' US Patent App.\ 16/090,453, published 2019.
\end{enumerate}


\section{\bf CONFERENCE TALKS}
\vspace{1em}\null
\begin{enumerate}
    \setlength\itemsep{0.8em}
    \item \textbf{JH Abel}, A~Asgari-Targhi, EB Klerman, and FJ~Doyle III,
``Designing a critical resetting protocol for achieving large phase shifts in humans,''
presented at Meeting of the Society for Research on Biological Rhythms (SRBR 2018), Amelia Island, Florida, USA, May 2018. (contributed talk)
    \item \textbf{JH Abel}, A~Chakrabarty, and FJ~Doyle~III, ``Nonlinear model predictive control for circadian entrainment using small-molecule pharmaceuticals,'' presented at IFAC World Congress 2017, Toulouse, France, July 2017. (contributed talk, proceedings listed above)
    \item \textbf{JH Abel}, ``Control of the Mammalian Circadian Oscillator,'' presented at International School and Conference on Network Science (NetSci) 2017, Indianapolis, Indiana, USA, June 2017. (invited talk)
    \item \textbf{JH Abel} and FJ Doyle~III, ``Identifying circadian drug targets for maintained oscillatory precision,'' presented at 2016 Meeting of the American Institute of Chemical Engineers (AIChE 2016), San Francisco, California, USA, November 2016. (contributed talk)
    \item \textbf{JH Abel}, K~Meeker, D~Granados-Fuentes, PC~St.~John, T~Wang, BB~Bales, ED~Herzog, LR~Petzold, and FJ~Doyle III,
``Inferring the functional resynchronization network in the suprachiasmatic nucleus,''
presented at Meeting of the Society for Research on Biological Rhythms (SRBR 2016), Palm Harbor, Florida, USA, May 2016. (contributed talk)
    \item \textbf{JH Abel} and LR~Petzold (jointly given). ``The effects of stochasticity on circadian rhythms,'' presented at Lorentz Center Workshop on Human Circadian Rhythms, Leiden, Netherlands, July 2015. (invited talk)
\end{enumerate}

\section{\bf INVITED LECTURES}
\vspace{1em}\null
\begin{enumerate}
    \setlength\itemsep{0.8em}
    \item \textbf{JH Abel}, 
``Reconciling AVP and VIP neurotransmission in the suprachiasmatic nucleus,'' presented at the Cognitive Rhythms Collaborative Retreat, MIT, March 2020.
    \item \textbf{JH Abel}, 
``Suprachiasmatic nucleus: a master circadian pacemaker in mammals,'' presented at MCB 186: Sleep and Circadian Clocks: From Biology to Public Health, Harvard University, February 2020.
    \item \textbf{JH Abel}, 
``Circadian oscillation in the malaria parasite: from genes to models,'' presented at Scientific Staff Meeting of the Division of Sleep and Circadian Disorders, Brigham and Women's Hospital, November 2019.
\item \textbf{JH Abel}, 
``Suprachiasmatic nucleus: a master circadian pacemaker in mammals,'' presented at MCB 186: Sleep and Circadian Clocks: From Biology to Public Health, Harvard University, February 2019.
\item \textbf{JH Abel}, 
``Controlling circadian rhtyhms,'' presented at Chronobiology and the Brain Seminar Series, Harvard Medical School, February 2018.
\item \textbf{JH Abel}, 
``Suprachiasmatic nucleus: a master circadian pacemaker in mammals,'' presented at MCB 186: Sleep and Circadian Clocks: From Biology to Public Health, Harvard University, February 2018.
\item \textbf{JH Abel}, 
``Suprachiasmatic nucleus: a master circadian pacemaker in mammals,'' presented at MCB 186: Sleep and Circadian Clocks: From Biology to Public Health, Harvard University, February 2017.
    \item \textbf{JH Abel}, 
``Modeling the Circadian Rhtythm,'' presented at CS 341: Systems Biology, Colby College, November 2015.
\end{enumerate}

\section{\bf TEACHING}
\vspace{1em}
\begin{tabular}{p{0.9in} p{5.1in}}
January 2020 & \textit{Introduction to Physiological Closed-Loop Control (HST S56)}, MIT\\
 & Role: Instructor\\
 & Taught 20-hour, 3-credit MIT course in control theory and its medical applications (in collaboration with three members of MIT NSRL). Approximately 20 graduate and undergraduate students.\\[0.5em]
Fall 2019 & \textit{Sleep (Gen Ed 1038)}, Harvard University\\
 & Role: Teaching Fellow\\
 & Instructors: Charles A. Czeisler, Frank A.J.L. Scheer\\[0.5em]
January 2019 & \textit{Introduction to Physiological Closed-Loop Control (HST S56)}, MIT\\
 & Role: Instructor\\
 & Taught 20-hour, 3-credit MIT course in control theory and its medical applications (in collaboration with four members of MIT NSRL). Approximately 20 graduate and undergraduate students.\\[0.5em]
Spring 2017 & \textit{Sleep and Circadian Clocks: Biology to Public Health (MCB 186)}, Harvard University\\
 & Role: Teaching Fellow\\
 & Instructors: Charles A. Czeisler, Frank A.J.L. Scheer\\[0.5em]
Fall 2014 & \textit{Analytical Methods in Chemical Engineering (CHE 132A)}, UCSB\\
 & Role: Teaching Assistant\\
 & Instructor: Baron Peters
\end{tabular}

\section{\bf UNDERGRADUATE STUDENTS MENTORED}
\vspace{1em}
\begin{tabular}{p{1.5in} p{3.5in} p{1.0in}}
Kimaya Lecamwasam & MIT/Wellesley Undergraduate Researcher & 2018 - \\
Shikha Sharma & Harvard University Summer Researcher & 2016\\
David McBride & UC Santa Barbara Undergraduate & 2015\\
Dustin Oakes & UC Santa Barbara HSAP & 2015\\
Amanda N. Luan & UC Santa Barbara Undergraduate (co- with P.C. St. John) & 2014\\
Jesse Starger & Tufts University Undergraduate & 2013
\end{tabular}

\section{\bf PROFESSIONAL ACTIVITIES}
\vspace{1em}
{\it Professional societies}\\[0.5em]
American Institute of Chemical Engineers (AIChE), Member\\
Society for Research on Biological Rhythms (SRBR), Member

%{\it Leadership positions}\\[0.5em]
%\begin{tabular}{p{0.5in} p{2in} p{3.5in}}
%2019 - &  Member, \\
%\end{tabular}

{\it Conference organizing}\\[0.5em]
\begin{tabular}{p{0.9in} p{5.1in}}
July 2020 & Chair, Organizer of Invited Session ``Precision Medicine Enabled by Automatic Control'' (with Lindsey Brown) at International Federation of Automatic Control (IFAC) World Congress\\[0.5em]
July 2017 & Chair, Organizer of Invited Session ``Optimal Control and Optimization of Biological Systems'' (with Steffen Waldherr) at International Federation of Automatic Control (IFAC) World Congress
\end{tabular}

{\it Peer reviewer}\\[0.5em]
Journal of Biological Rhythms\\
International Federation of Automatic Control (IFAC) World Congress\\
IEEE International Conference on Biomedical and Health Informatics\\
IEEE Engineering in Medicine and Biology Society (EMBS) Conference\\
PeerJ Computer Science



{\it Computing skills}\\
\begin{itemize}
\item Languages: Python (including numpy, scipy, scikit-learn, pandas, cython), Matlab, some use of C++, R, and Unix shell scripts.
\item Algorithms: Experience in methods for numerical solutions to ordinary and partial differential equations,
      stochastic simulation algorithms, linear/nonlinear programming, evolutionary algorithms, and artificial neural networks.
\item Applications: Vim, Wolfram Mathematica, \LaTeX, Stata, common Microsoft database, spreadsheet, and presentation software.
\item Operating Systems:  Unix/Linux, OSX.\\
\end{itemize}


\end{resume}
\null\hfill Current as of: \today.
\end{document}

\section{Miscellaneous}

\textbf{Erd\H{o}s Number (4)}

\textbf{A.B. Barak (Erd\H{o}s 1)} and P. Erd\H{o}s, ``On the maximal number of strongly independent vertices in a random acyclic directed graph,'' \textit{SIAM Journal of Algebraic Discrete Methods}, vol. 5, no. 4, 1984.\\
A.B. Barak and \textbf{P.E. Saylor (Erd\H{o}s 2)}, ``A symmetric factorization procedure for the solution of elliptic boundary value problems,'' SIAM \textit{Journal of Numerical Analysis}, vol. 10, 1973.\\
S.F. Ashby, S.L. Lee, \textbf{L.R. Petzold (Erd\H{o}s 3)}, P.E. Saylor, and E. Seidel, ``Computing spacetime curvature via differential algebraic equations,'' \textit{Applied Numerical Mathematics}, vol. 20, no. 1-2, 1996.\\
\textbf{J.H. Abel (Erd\H{o}s 4)}$^\dagger$, K. Meeker$^\dagger$, D. Granados-Fuentes, P.C. St. John, T. Wang, B.B. Bales, F.J. Doyle III, E.D. Herzog, and L.R. Petzold,
``Functional network inference of the suprachiasmatic nucleus,''
{\it Proceedings of the National Academy of Sciences of the USA}, vol. 113, no. 16, 2016.

\end{resume}


\hfill Current as of: \today
\end{document}



